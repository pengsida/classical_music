% !TeX spellcheck = en_US
%% 字体:方正静蕾简体
%%		 方正粗宋
\documentclass[a4paper,left=2.5cm,right=2.5cm,11pt]{article}

\usepackage[utf8]{inputenc}
\usepackage{fontspec}
\usepackage{cite}
\usepackage{xeCJK}
\usepackage{indentfirst}
\usepackage{titlesec}
\usepackage{longtable}
\usepackage{graphicx}
\usepackage{float}
\usepackage{rotating}
\usepackage{subfigure}
\usepackage{tabu}
\usepackage{amsmath}
\usepackage{setspace}
\usepackage{amsfonts}
\usepackage{appendix}
\usepackage{listings}
\usepackage{xcolor}
\usepackage{geometry}
\setcounter{secnumdepth}{4}
\usepackage{mhchem}
\usepackage{multirow}
\usepackage{extarrows}
\usepackage{hyperref}
\titleformat*{\section}{\LARGE}
\renewcommand\refname{参考文献}
\renewcommand{\abstractname}{\sihao \cjkfzcs 摘{  }要}
%\titleformat{\chapter}{\centering\bfseries\huge\wryh}{}{0.7em}{}{}
%\titleformat{\section}{\LARGE\bf}{\thesection}{1em}{}{}
\titleformat{\subsection}{\Large\bfseries}{\thesubsection}{1em}{}{}
\titleformat{\subsubsection}{\large\bfseries}{\thesubsubsection}{1em}{}{}
\renewcommand{\contentsname}{{\cjkfzcs \centerline{目{  } 录}}}
\setCJKfamilyfont{cjkhwxk}{STXingkai}
\setCJKfamilyfont{cjkfzcs}{STSongti-SC-Regular}
% \setCJKfamilyfont{cjkhwxk}{华文行楷}
% \setCJKfamilyfont{cjkfzcs}{方正粗宋简体}
\newcommand*{\cjkfzcs}{\CJKfamily{cjkfzcs}}
\newcommand*{\cjkhwxk}{\CJKfamily{cjkhwxk}}
\newfontfamily\wryh{Microsoft YaHei}
\newfontfamily\hwzs{STZhongsong}
\newfontfamily\hwst{STSong}
\newfontfamily\hwfs{STFangsong}
\newfontfamily\jljt{MicrosoftYaHei}
\newfontfamily\hwxk{STXingkai}
% \newfontfamily\hwzs{华文中宋}
% \newfontfamily\hwst{华文宋体}
% \newfontfamily\hwfs{华文仿宋}
% \newfontfamily\jljt{方正静蕾简体}
% \newfontfamily\hwxk{华文行楷}
\newcommand{\verylarge}{\fontsize{60pt}{\baselineskip}\selectfont}  
\newcommand{\chuhao}{\fontsize{44.9pt}{\baselineskip}\selectfont}  
\newcommand{\xiaochu}{\fontsize{38.5pt}{\baselineskip}\selectfont}  
\newcommand{\yihao}{\fontsize{27.8pt}{\baselineskip}\selectfont}  
\newcommand{\xiaoyi}{\fontsize{25.7pt}{\baselineskip}\selectfont}  
\newcommand{\erhao}{\fontsize{23.5pt}{\baselineskip}\selectfont}  
\newcommand{\xiaoerhao}{\fontsize{19.3pt}{\baselineskip}\selectfont} 
\newcommand{\sihao}{\fontsize{14pt}{\baselineskip}\selectfont}      % 字号设置  
\newcommand{\xiaosihao}{\fontsize{12pt}{\baselineskip}\selectfont}  % 字号设置  
\newcommand{\wuhao}{\fontsize{10.5pt}{\baselineskip}\selectfont}    % 字号设置  
\newcommand{\xiaowuhao}{\fontsize{9pt}{\baselineskip}\selectfont}   % 字号设置  
\newcommand{\liuhao}{\fontsize{7.875pt}{\baselineskip}\selectfont}  % 字号设置  
\newcommand{\qihao}{\fontsize{5.25pt}{\baselineskip}\selectfont}    % 字号设置 

\usepackage{diagbox}
\usepackage{multirow}
\boldmath
\XeTeXlinebreaklocale "zh"
\XeTeXlinebreakskip = 0pt plus 1pt minus 0.1pt
\definecolor{cred}{rgb}{0.8,0.8,0.8}
\definecolor{cgreen}{rgb}{0,0.3,0}
\definecolor{cpurple}{rgb}{0.5,0,0.35}
\definecolor{cdocblue}{rgb}{0,0,0.3}
\definecolor{cdark}{rgb}{0.95,1.0,1.0}
\lstset{
	language=C,
	numbers=left,
	numberstyle=\tiny\color{white},
	showspaces=false,
	showstringspaces=false,
	basicstyle=\scriptsize,
	keywordstyle=\color{purple},
	commentstyle=\itshape\color{cgreen},
	stringstyle=\color{blue},
	frame=lines,
	% escapeinside=``,
	extendedchars=true, 
	xleftmargin=0em,
	xrightmargin=0em, 
	backgroundcolor=\color{cred},
	aboveskip=1em,
	breaklines=true,
	tabsize=4
} 

\newfontfamily{\consolas}{Consolas}
\newfontfamily{\monaco}{Monaco}
\setmonofont[Mapping={}]{Consolas}	%英文引号之类的正常显示,相当于设置英文字体
\setsansfont{Consolas} %设置英文字体 Monaco, Consolas,  Fantasque Sans Mono
\setmainfont{Times New Roman}

\setCJKmainfont{华文中宋}


\newcommand{\fic}[1]{\begin{figure}[H]
		\center
		\includegraphics[width=0.8\textwidth]{#1}
	\end{figure}}
	
\newcommand{\sizedfic}[2]{\begin{figure}[H]
		\center
		\includegraphics[width=#1\textwidth]{#2}
	\end{figure}}

\newcommand{\codefile}[1]{\lstinputlisting{#1}}

\newcommand{\interval}{\vspace{0.5em}}

\newcommand{\tablestart}{
	\interval
	\begin{longtable}{p{2cm}p{10cm}}
	\hline}
\newcommand{\tableend}{
	\hline
	\end{longtable}
	\interval}

% 改变段间隔
\setlength{\parskip}{0.2em}
\linespread{1.1}

\usepackage{lastpage}
\usepackage{fancyhdr}
\pagestyle{fancy}
\lhead{\space \qquad \space}
\chead{马勒第三交响曲 \qquad}
\rhead{\qquad\thepage/\pageref{LastPage}}
\begin{document}

\tableofcontents

\clearpage

\section{简介}
	马勒的《第三交响曲》,D小调,作于1895-1896年,表现的是马勒的自然观。马勒原定此曲标题为《夏日正午之梦》。

\section{乐章段落}
\subsection{第一部分}
	\begin{itemize}
		\item[1.] 引子,牧神潘在乐曲开头的哀乐声中入睡,标志着夏日的来临(酒神巴克斯的行列)。
	\end{itemize}

\subsection{第二部分}
	\begin{itemize}
		\item[2.]草原的花朵告诉我
		\item[3.]森林的动物告诉我
		\item[4.]人类告诉我
		\item[5.]天使告诉我
		\item[6.]爱情告诉我
	\end{itemize}

\section{乐章赏析}
\subsection{第一乐章}
	D小调,指示“强有力而决然地”,扩大的奏鸣曲式。8把圆号有力地表现第一主题,铜管与打击乐以进行曲节奏加强,加入小号的信号曲动机,表示夏天接 近森林。圆号再表现由第一主题动机发展而成的第二主题,小号承接,圆号反复后,进入第三主题部,表现拒绝苏醒的牧神。牧神的苏醒在大自然类似小鸟、动物的 叫声衬托下非常动人。然后单簧管奏第四主题,小提琴承接它趋于平静进入呈示部小结尾。发展部以圆号表现的第二主题开始,加入小号的信号曲动机,平静之后由 长号移自英国管,在平静中进行第三主题。然后再出现第四主题与第三主题的重合,回到进行曲风格,平静下来后双簧管再奏第三主题,紧接着与经过部旋律的第一 主题作对位性结合,达到辉煌顶点后进入再现部。再现部比呈示部缩小很多,按顺序再现各主题后,圆号出现第一主题,进入呈示部一样的小结尾,造成高潮。

\subsection{第二乐章}
	小步舞曲速度,A大调,优雅的洛可可风格,扩大的三段体,即把中段出现两次。以双簧管表现主要主题,各件乐器一一发展。第一中段是升 F小调,长笛与中提琴奏乡愁感的主旋律。第二中段由双簧管与单簧管开始,然后长笛与中提琴第三次再现主题,最后以泛音的高音和弦来结束。

\subsection{第三乐章}
	悠闲的谐谑曲,C小调,指示为“不急速地”。这一乐章据《少年的魔角》中的《夏末》的歌词而作,歌词大意为:“杜鹃掉进柳树的洞穴里死了,夜莺在翠 绿的枝头啼啭,将让我们快乐……”自由的三段体,先由单簧管奏出主麒其它木管乐器奏出如小鸟啼叫一样的音形。天色大亮,小提琴以新的动机加入。在经过短暂 的经过句后,小号奏出信号型旋律,邮号承接,营造出神秘的森林气氛。再一次缓慢地奏出主要主题后,突然变成华丽的氛围,传来奥地利军队的旋律,然后又以邮 号来引向宁静。

\subsection{第四乐章}
	D大调,“极为缓慢、神秘地”。在平静的演奏后,女低音独唱,歌词为尼采《查拉图斯特拉如是说》第四部醉歌后查拉图斯特拉的轮唱。歌词大意是: “啊,人类,请注意,深夜说了些什么,我睡着了,从深沉的睡梦中醒来,世界比白天所想的更为深沉,啊,人类,非常的深沉,苦恼是非常的深沉,快乐比伤心更 为深沉,苦恼说灭亡吧,然而所有的快乐,却企求深远的永恒。”

\subsection{第五乐章}
	F大调,指示为“以活泼的速度,表现要大胆。” 先以童声合唱模仿钟声而反复“宾、邦”,然后女声合唱、女声独唱,歌词为《少年的魔角》中《3个天使在唱快乐的歌》。歌词大意为:“3个天使在唱快乐的 歌,在天上,那是快乐而幸福的音响,于是他们快乐地欢笑,说彼得无罪,主耶稣就坐在餐桌旁,与12圣徒一起享用最后的晚餐,耶稣说,你在干什么,我看到, 你在流泪,难道我不该流泪,你慈悲的上帝,我已犯了十诫,我跚跚独行泪湿衣衫,你不该哭,啊,来吧,请怜悯我,如果你犯了十诫,快跪下向上帝祈祷,只爱永 恒的上帝,这样你才能懂得天国的欢乐,天国之域已为你彼得做了准备,通过基督,为了拯救所有的人。”最后,以女声与童声一起合唱“宾、邦”而结束。

\subsection{第六乐章}
	D大调,指示为“缓慢、平静、充满感情地”,自由的回旋曲式。开始以弦乐幽静地表现主要主题,对位极为优美。加入木管后以小调表现副主题,以新的对 位,由小提琴再现主要主题,加上第一乐章第一主题动机,形成高潮。然后主题又宁静地再现,加强力度后加上第一乐章小结尾动机,达到雄壮的高潮。再以铜管出 现主要主题,纠缠着副主题而发展,最后以光明而结尾。这个乐章像是从对天使的凝视开始,最后是讴歌爱而结束。
	

\end{document}
